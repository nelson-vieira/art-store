\documentclass[conference]{IEEEtran}
\IEEEoverridecommandlockouts
% The preceding line is only needed to identify funding in the first footnote. If that is unneeded, please comment it out.
\usepackage{cite}
\usepackage{amsmath,amssymb,amsfonts}
\usepackage{algorithmic}
\usepackage{textcomp}
\usepackage{xcolor}
\usepackage{colortbl}
\usepackage[rightcaption]{sidecap}
\usepackage{wrapfig}
\usepackage{adjustbox}
\usepackage{siunitx}
\usepackage{hyperref}
\def\UrlBreaks{\do\/\do-}
\usepackage{graphicx} %package to manage images
\graphicspath{ {./images/} }


\def\BibTeX{{\rm B\kern-.05em{\sc i\kern-.025em b}\kern-.08em
    T\kern-.1667em\lower.7ex\hbox{E}\kern-.125emX}}
\begin{document}

\title{Online Store Application: Art Store+}

\author{\IEEEauthorblockN{1\textsuperscript{st} Tomás Marcos}
\IEEEauthorblockA{\textit{Faculdade de Ciências Exatas e da Engenharia} \\
\textit{Universidade da Madeira}\\
Funchal, Portugal \\
2037017@student.uma.pt}
\and
\IEEEauthorblockN{2\textsuperscript{nd} Nelson Vieira}
\IEEEauthorblockA{\textit{Faculdade de Ciências Exatas e da Engenharia} \\
\textit{Universidade da Madeira}\\
Funchal, Portugal \\
2080511@student.uma.pt}
\and
\IEEEauthorblockN{2\textsuperscript{nd} Luís Olim}
\IEEEauthorblockA{\textit{Faculdade de Ciências Exatas e da Engenharia} \\
\textit{Universidade da Madeira}\\
Funchal, Portugal \\
2034717@student.uma.pt}
}

\maketitle

\begin{abstract}
A água é um recurso precioso, considerado um dos bens essenciais para a vida.
No entanto, cada vez mais, ouve-se que é um recurso escasso e que rapidamente
está a se esgotar. A água é utilizada para muitas atividades, sejam elas industriais,
comerciais ou de lazer. Existem muitas iniciativas que pretendem reduzir o
consumo e o desperdício de água. Pretendemos explorar um sistema de rega
inteligente que utilize sensores de forma a reduzir a quantidade de água que
é utilizada. \\
\end{abstract}

\begin{IEEEkeywords}
IoT, computação ubíqua, rega inteligente.
\end{IEEEkeywords}

\section{Introdução}

\IEEEPARstart{A} sustentabilidade global não será alcançada sem garantir a
disponibilidade de água preciosa para todos os consumidores. Apesar de ser um
dos principais objetivos da agenda da UN2030 \cite{un2015agenda} para o desenvolvimento
global sustentável, a atual escassez de água está a crescer rapidamente e
afetando um número crescente de consumidores de água residencial, comercial,
industrial e agrícola em todo o mundo \cite{mishra2021water}. Espera-se que a
procura global da água suba 55\%, enquanto atualmente, cerca de 25\% das grandes
cidades estão a passar por alguns níveis de stress hídrico \cite{josefine2021differentiated}.

As mudanças climáticas, secas graves, crescimento populacional, aumento da
procura e má administração durante as últimas décadas enfatizaram ainda mais
os recursos escassos da água doce em todo o mundo e resultaram numa grave
escassez de água para cerca de 4 bilhões de pessoas, pelo menos um mês
anualmente \cite{jafari2018assessing} \cite{unicef2019progress} \cite{orimoloye2021spatial}. \cite{salehi2022global}

Um dos setores de atividade humana que tem maior consumo dos recursos hídricos
é a agricultura, "aproximadamente 100 vezes mais do que o uso pessoal é consumida
pela alimentação e agricultura e quase 70\% das águas fluviais e subterrâneas
são utilizadas na irrigação". \cite{nawandar2019iot} Apenas 17\% das terras
agrícolas, de todo o mundo são irrigadas. Apesar de pôr todo o mundo, se verificar
um aumento de terrenos irrigados, a área irrigada per capita tem estado a diminuir
desde 1990 devido ao rápido crescimento global. \cite{pimentelwater}

Existem, atualmente, várias tecnologias de irrigação, sendo as mais comuns
a irrigação por inundação e irrigação por aspersão. Outros métodos mais
focados, como a irrigação gota-a-gota têm maior eficiência hídrica. Esta
técnica usa uma quantidade inferior de água, entre 30\% a 50\%, quando
comparada com uma técnica de irrigação superficial. \cite{pimentelwater}

Várias iniciativas foram tomadas para ajudar a minimizar o desperdício
de água neste setor, no entanto, não aparentam ter muito sucesso,
ou não são apelativas, devido aos elevados custos associados.
Os sensores comerciais para sistemas destinados à agricultura e à sua
irrigação são muito caros, impossibilitando aos pequenos agricultores
a implementação deste tipo de sistema nas suas explorações. No entanto,
os fabricantes oferecem atualmente sensores de baixo custo que podem
ser ligados a aparelhos também de baixo custo para implementar sistemas de baixo custo para a gestão da
irrigação e monitorização agrícola. Além disso, devido ao interesse em
sensores de baixo custo para monitorizar a agricultura e a água,
novos sensores de baixo custo estão a ser propostos em vários estudos. \cite{garcia2020iot}

Portugal Continental, no mês de abril de 2022, encontrava-se numa situação de seca, nos
quais 81,9\% do território encontrava-se em seca moderada e 17,9\% do território em seca severa,
segundo o jornal Expresso \cite{expresso}.

Segundo o Instituto Português do Mar e da Atmosfera (IPMA),
a quantidade de precipitação até o dia 15 de abril de 2022, teve um valor médio inferior ao valor normal
quando comparado com anos anteriores, correspondente a 38\%. Devido à redução da precipitação,
a percentagem de água no solo diminuiu em quase todo o país com valores inferiores a 20\%, com alguns locais
a atingirem o ponto de emurchecimento. O boletim também contém um gráfico, demonstrado na figura \ref{fig:ipmagraph}
que mostra a percentagem de território do país sem situações de seca. \cite{ipmaboletim}

\begin{figure}[h]
    \centering
    \includegraphics[scale=0.5]{grafico-seca-portugal.png}
    \caption{Percentagem do território de Portugal Continental por classe do índice PDSI em situações de seca anteriores em abril (2022 até dia 15)}
    \label{fig:ipmagraph}
\end{figure}

Por estes motivos é importante gerir o consumo de água no nosso dia a dia,
portanto, o que propomos é um sistema de rega inteligente que faz a
medição da humidade do solo e rega as plantas apenas durante o tempo
necessário poupando o gasto desnecessário da água de rega.

\section{Trabalhos Relacionados}

Segundo um estudo realizado por García et al. \cite{garcia2020iot}, existem 178 artigos
relacionados com  "IoT irrigation, IoT irrigation system, and smart
irrigation", escritos em inglês, no período de entre os
anos de 2014 e 2019, inclusive, dos quais 106 artigos estão relacionados com a
utilização de sensores para monitorizar o estado do solo. Destes 106 artigos
estudados, todos os artigos abordam a humidade do solo, 9 discutem a temperatura
do solo, 4 exploram o ph do solo e 3 mencionam os nutrientes presentes no solo.
Para determinar a seca causada por anomalias nas águas superficiais,
foi utilizado o Standardized Precipitation Evapotranspiration Index (SPEI).
Estes índices e a informação recolhida a partir de sensores que monitorizam o ambiente,
o solo e a água podem ser utilizados para determinar o estado atual da água e a
possibilidade de cobrir todas as necessidades de água doce. Os países com maiores
fundos já implementam sistemas de gestão e reutilização da água com o objetivo de
otimizar a utilização da água e reduzir o impacto ambiental causado pela utilização
de grandes quantidades de água. No entanto, alguns países podem considerar que estas
soluções são dispendiosas.
Os sensores comerciais para sistemas destinados à agricultura e à sua irrigação
são muito caros, impossibilitando aos pequenos agricultores a implementação deste
tipo de sistema nas suas explorações. No entanto, os fabricantes oferecem atualmente
sensores de baixo custo que podem ser ligados a nós para implementar sistemas de baixo
custo para a gestão da irrigação e monitorização agrícola. Além disso, devido ao interesse
em sensores de baixo custo para monitorizar a agricultura e a água, estão a ser propostos
novos sensores de baixo custo em investigações tais como um sensor de monitorização do
stress hídrico foliar \cite{daskalakis2018}, um sensor de humidade do solo de vários níveis composto
por anéis de cobre colocados ao longo de um tubo de PVC \cite{guruprasadh2017intelligent},
um sensor de monitorização da salinidade da água feito com bobinas de
cobre \cite{parra2013low} ou um sensor de turbidez da água
feito com emissores e recetores de chumbo colorido e infravermelho \cite{sendra2013low}.

Nas atividades agrícolas que utilizam água, também conhecidas
como agricultura irrigada, há diferentes maneiras de distribuir a água.
As diferentes opções apresentam eficiência diferente e, em alguns casos,
uma forma específica deve ser utilizada para uma cultura específica.
As formas específicas de irrigação têm uma grande variedade, mas podemos
dividi-las nas seguintes categorias: Podemos considerar a forma de
distribuição da água:

\begin{itemize}
\item irrigação por inundação;
\item irrigação por aspersão;
\item irrigação gota a gota;
\item irrigação por nebulização.
\end{itemize}

Quanto à existência de sistemas de deteção, podemos ter:

\begin{itemize}
\item irrigação sem qualquer consideração, quando a quantidade de água não é calculada ou estimada;
\item irrigação programada, quando a água é fornecida de acordo com as necessidades estimadas num período do ano;
\item irrigação ad hoc, quando a quantidade de água é calculada com base nas medições dos sensores.
\end{itemize}

A grande maioria dos trabalhos incluídos propõe a utilização de
bombas e válvulas para distribuir a água em conjunto com sensores para medir os
parâmetros ambientais, a fim de calcular as necessidades de água. Dos 89
artigos avaliados, 83 incluem informações claras sobre o
sistema de irrigação proposto, os outros seis apenas mencionam que incluem
atuadores para irrigação. Estes 83 incluem diferentes níveis
de detalhe, existem 49 documentos que apenas indicam que existem motores/bombas
no seu sistema (40 documentos) ou válvulas (nove documentos) sem mais detalhes.
Dos artigos que oferecem mais detalhes, 19 incluem aspersores (o sistema
mais utilizado) \cite{gonzalez2018iot, ahmed2016intelligation, yusuf2005information, cambra2017iot, arvind2017automated, ammour2018factory, singh2019iot, wu2016secure, solanki2017conceptual, wasson2017integration, johar2018iot, ryu2015design, reche2014smart, chieochan2017internet, arumugam2018internet, boonchieng2018smart, rawal2017iot, guo2015design, khattab2016design},
oito utilizam irrigação por gotejamento \cite{daskalakis2018uw, nawandar2019iot, barkunan2019smart, sivaprasath2016arduino, kumar2017internet, kodali2016iot, abidin2015web, banumathi2017android}, dois propõem a
utilização de pulverizadores \cite{mechsy2017mobile}, e os restantes utilizam
um sistema de irrigação muito específico (robts \cite{rahul2018iot}, pivot,
pistola de chuva \cite{vasu2017intelligent} ou pode ser aplicada a múltiplos
sistemas \cite{agale2017automated}). Em conjunto com o sistema de irrigação principal,
três artigos propõem o uso de um sistema de nebulização \cite{chieochan2017internet, boonchieng2018smart, kodali2016iot} e dois artigos
propõem o uso de fertirrigação nos seus sistemas \cite{arumugam2018internet, abidin2015web}.
Dos artigos que mencionam o tipo de sensor utilizado, o sensor mais popular é o
e YL69 (SparkFun Electronics, Niwot, CO, USA). Este sensor tem um baixo custo e
foi criado para operar especificamente com o Arduino. \cite{garcia2020iot}

Muitos sistemas de rega inteligente têm por base sensores de humidade so solo, como
é o caso dos sistemas que analisamos de seguida.

\subsection{Sistema proposto por Abbas et al.}

No estudo realizado por Abbas et al. \cite{abbas2014smart}, os autores propõem
um sistema de irrigação inteligente no qual utilizaram uma rede de sensores sem
fios para detetar a humidade no solo. Um dos focos do estudo proposto foi a medição
do tempo de resposta e a capacidade do sistema identificar a capacidade de
retenção de água do tipo de solo no qual os sensores estavam localizados.

Como o solo argiloso pode manter a água por longos períodos. Por outro lado,
o solo arenoso não pode manter a água enquanto o solo argiloso for argiloso.
Por conseguinte, a água não pode ser mantida, uma vez que a medição da
humidade do solo no solo argiloso ou no solo de vaso é feito de 2 em 2 horas,
o mesmo processo tem de ser realizado em tempo mais curto no caso de vasos
ou campos de terra arenosa. Mas a resposta da humidade do solo argiloso
à água é mais lenta do que a resposta da humidade do solo arenoso para a água.
Assim, o tempo de espera de deteção após a irrigação do solo
arenoso espera -se que seja menor que o correspondente tempo para o solo argiloso. Obviamente,
os parâmetros de tempo devem ser alterado com base não só no tipo de solo mas
também no tipo de planta porque os requisitos de irrigação para as plantas
diferem de cada outros. Também as estações desempenham um grande papel no
processo de irrigação. No verão, as plantas precisam de ser regadas com mais
frequência do que no inverno. Tendo em mente a saúde dos pacotes será muito
importante no caso das plantas levedadas em especial, elas crescem a uma
distância relativamente alta do solo. Nesses casos, os nós poderão ser
levantados em paus altos. Há casos em que o alojamento ou cobertura
para os nós é necessária para evitar os seus danos; por exemplo
chuva forte, queda de granizo, cobertura de neve, e raios ardentes
do sol em longos dias de verão. \cite{abbas2014smart}

\subsection{Sistema proposto por Goap et al.}

A humidade do solo é um parâmetro crítico para o desenvolvimento de um
sistema de irrigação inteligente. A humidade do solo é afetada por uma
série de variáveis ambientais, por exemplo, temperatura do ar,
humidade do ar, UV, temperatura do solo, etc. Com o avanço das tecnologias,
a precisão das previsões meteorológicas melhorou significativamente e
os dados meteorológicos previstos podem ser utilizados para prever as
alterações da humidade do solo. Este artigo propõe uma arquitetura
de irrigação inteligente baseada em IoT juntamente com uma abordagem
baseada na aprendizagem de máquinas híbridas para prever a humidade do solo.
O algoritmo proposto utiliza os dados dos sensores do passado recente e
os dados meteorológicos previstos para a previsão da humidade do solo
dos próximos dias. O valor previsto da humidade do solo é melhor em
termos da sua precisão e taxa de erro. Além disso, a abordagem de
previsão é integrada num protótipo de sistema autónomo. O protótipo
do sistema é rentável, uma vez que se baseia nas tecnologias de padrão aberto.
O modo automático torna-o um sistema inteligente e pode ser ainda
mais personalizado para aplicação de cenários específicos. \cite{goap2018an}

\subsection{Sistema proposto por Premkumar e Sigappi}

O modelo de aprendizagem proposto por Premkumar e Sigappi \cite{premkumar2022iot} para irrigação é implementado num protótipo
de sistema IoT que tem quatro componentes: (i) Camada de nó de borda -
Esta camada é constituída por sensores, atuador, e dois microcontroladores.
Nesta camada, o nó de borda adquire os dados dos sensores do ambiente e controla
o atuador para acionar as bombas de água para iniciar a irrigação.
(ii) Camada de servidor de borda - Esta camada consiste em Raspberry Pi que
atua como servidor de borda e capaz de processamento multitarefa. Aqui, o
servidor de borda controla os nós de borda para enviar sinal e receber dados
a intervalos regulares de tempo. Está também ligado ao cloud server para receber
o modelo de aprendizagem de máquinas desenvolvido e treinado para ser implantado
e tomar decisões de irrigação para controlar os nós de borda.
(iii) camada de serviço de borda - Esta camada é implantada no servidor de
borda e é responsável pelo controlo de todo o sistema através de um painel de
controlo web desenvolvido. O painel de bordo tem dados de alimentação ao vivo,
controlo dos nós de borda, e acesso aos serviços de nuvem. Esta camada de
serviço tem também o controlo de acesso do modelo de aprendizagem da máquina
proposto. (iv) Camada de servidor de nuvem - Esta camada composta por serviços
de nuvem e armazenamento de nuvens onde o seu papel é treinar o modelo de
aprendizagem da máquina e armazenar os dados na base de dados. Envia o modelo
proposto treinado para o servidor de borda para a tomada de decisões relativas
à programação da irrigação. O proposto sistema de irrigação inteligente baseado
em IoT inclui cinco componentes principais: módulo implantado no campo,
interface baseada na Web, entrada de tempo Web API, humidade do solo
mecanismo de previsão, e modelo de comunicação de borda.


\section{Métodos e Metadologias}

O trabalho descrito neste artigo pretende responder a algumas questões que foram
levantadas após alguma investigação sobre soluções já existentes no que diz respeito
a sistemas de rega. O sistema proposto permite poupar água? Qual a quantidade de água
que é possível poupar?  Qual é o custo associado à integração de sensores num sistema
de rega convencional? Em comparação com um sistema de rega convencional,
qual a poupança que um sistema de rega inteligente proporciona?

\subsection{O nosso sistema}

O sistema de rega inteligente que propomos faz uso do Arduino MKR 1000 WiFi,
de um sensor de humidade do solo, de uma breadboard, de vários LEDs, uma
resistência de \SI{200}{\ohm} e um servo motor. O servo motor que estamos a
usar neste sistema serve apenas para efeitos de demonstração, pois
este sistema teria que ter uma bomba de água para fazer a rega
das plantas de forma automática, da forma como está neste momento, este
sistema de rega requer a intervenção humana para o seu funcionamento, mas
esta não era a intenção da sua criação.
Decidimos usar este modelo do Arduino pela ligação
à rede por Wi-fi que possui, o que nos permite analisar o sistema sem termos de
estar no local da instalação, o que podemos fazer através do Arduino Cloud,
que nos envia os resultados do sensor que estamos a usar. Caso o utilizador
queira ver o estado da humidade do solo, pode perceber pelos LEDs que usamos no sistema,
estes LEDs mostram um feedback visual do estado atual do solo, decidimos representar este
feedback com 5 LEDs de várias cores que vão desde o verde até ao vermelho, portanto
se o LED verde estiver acesso quer dizer que o solo está suficientemente humido, se
o LED laranja estiver acesso quer dizer que solo requer um pouco de água e se o LED
vermelho estiver acesso quer dizer que o solo está seco, por isso tem que ser regado.

O dispositivo Arduino, como mostra a figura \ref{fig:circuit}, que usamos é o
modelo MKR 1000 WiFi, pois é um modelo com capacidade wifi, o que facilita na
transmissão dos dados para o utilizador que poderá vê-los no seu smartphone,
esta interação com o smartphone não foi criada, mas é uma possibilidade, bem
como a utilização de um Raspberry Pi para fazer o tratamento da informação
recebida pelo Arduino.

O sensor de humidade, ilustrado pela figura \ref{fig:circuit}, é um
sensor normal para esta função, que tem por base valores entre 0 e 1023, serão usados valores
incrementais entre os valores mínimo e máximo para fazer uma distinção do grau
de escassez do solo, a partir destes valores base associamos uma percentagem
que irá corresponder à humidade do solo, a utilização de percentagem é
mais user-friendly e consequentemente o sistema dará um feedback mais útil
para o utilizador. Também poderia ser usado um Raspberry Pi para guardar dados do sensor.

Testou-se o sistema com dois sensores de humidade, mas não se observou variações de valores significativas,
sendo que a distinção dos sensores considerou-se não ser relevante.

\begin{figure}
    \centering
    \includegraphics[scale=0.5]{soil-moisture-circuit-schema.png}
    \caption{Esquema do Circuito}
    \label{fig:circuit}
\end{figure}

\subsubsection{Preço do sistema}

Na tabela \ref{pricetable} podemos verificar o custo do sistema,
isto inclui apenas os custos associados aos sensores e equipamento
estritamente necessário à criação deste sistema, portanto custo associados
a produtos ou serviços externos não são considerados para esta análise,
como, por exemplo, o custo de água gasta, pois estes valores variam ao longo
do tempo e por região e seria difícil fazer uma estimativa para tal.

\begin{table}[ht]
\centering
\small
\begin{tabular}{|c|c|c|c|c|}
    \hline
    \rowcolor{gray}
    \color{white}Item & \color{white}Preço \\
    \hline
    Arduino MKR 1000 WiFi & 30,70 € \\
    \hline
    Sensor de humidade & 5,40 € \\
    \hline
    Servo motor & 5,10 € \\
    \hline
    LEDs & 13 € (Pack) \\
    \hline
    Resistência & 6 € (Pack de 100) \\
    \hline
    Fios e outros cabos e Breadboard & 7,00 \\
    \hline
    \rowcolor{gray}
    \color{white}Total & \color{white}67,20 € \\
    \hline
\end{tabular}
\vspace{1em}
\caption{Custo do material utilizado}
\label{pricetable}
\end{table}

Como podemos verificar o custo total deste sistema, ou de um sistema de
rega inteligente similar, não é muito elevado. Apesar de ser difícil estimar
um valor concreto para o custo de água gasta podemos assumir que seria um custo
muito mais baixo do que é no momento para um utilizador que não use um sistema de
rega inteligente, pois este tipo de sistema está desenhado para poupar água.
Comparando com alguns sistemas de rega inteligente que existem no
mercado \cite{amazonOrbit} \cite{amazonNetro}, o nosso sistema é mais barato e
atendendo ao facto de que os preços descritos na tabela \ref{pricetable} incluem packs de
produtos e a utilização do servo motor neste sistema é para efeitos de simulação
de uma bomba de água, o preço real do sistema seria ainda mais barato, o preço mais
baixo poderia ser por volta de 40 €, o que tornaria este sistema atraente para
potenciais consumidores interessados na sustentabilidade e na redução da sua pegada
ambiental.

\subsection{Testes ao sistema}

Realizamos vários testes, tanto num ambiente controlado como em ambiente real, desde
a construção do circuito em simulação até ao momento de entrega final. Começamos
por fazer testes simulados, para isso usamos a ferramenta Tinkercad \cite{tinkercad} onde
criamos o circuito e o código associado ao circuito que implementa a lógica
que o sistema deve seguir, neste ambiente de simulação é possível testar
o sistema de forma a verificar se o código criado faz realmente aquilo
que é pretendido.

Num dos testes do primeiro protótipo do sistema, no qual apenas se tinha o microcontrolador
Arduino Uno e o sensor de humidade, foi possível observar que o valor máximo obtido pelo sensor
era de 1023, quando o recipiente com terra continha demasiada água. Este teste, por ainda
não se ter implementado todo o circuito considerou-se ser um teste preliminar.

\begin{figure}[h]
    \centering
    \includegraphics[scale=0.06]{sistema-de-rega-teste-ambiente-real.jpg}
    \caption{Teste realizado}
    \label{fig:teste}
\end{figure}

Posteriormente, realizou-se vários testes no jardim de um dos discentes, como mostra a figura \ref{fig:teste},
no qual os resultados obtidos num dos testes, podem ser verificados pelo gráfico da figura \ref{fig:graphic}.

Como se pode observar pelo gráfico, os valores iniciais estão comprimidos entre os 280 e os 300,
havendo uma descida abrupta logo após se ter iniciado o teste. Isto deve-se ao facto de se ter começado
o processo de irrigação. O solo por estar seco começou a absorver água e o sensor perdeu alguma estabilidade.
Após, cerca de 1 minuto, verificou-se uma subida constante no valor da humidade do solo até se atingir
um valor máximo de 400. Após se atingir este pico, os valores estabilizaram-se, havendo uma pequena variação
nos valores da humidade. Deixou-se de regar a planta e observou-se que os valores da humidade, começaram
a diminuir.

\begin{figure}
    \centering
    \includegraphics[scale=0.5]{humidity-test-graph.png}
    \caption{Gráfico dos resultados do teste realizado}
    \label{fig:graphic}
\end{figure}

Os resultados obtidos revelaram ser muito abaixo dos valores esperados. Sendo que nos testes preliminares
foi possível obter valores muito mais elevados, tal que foi possível testar o limite máximo do sensor. Seria
expectável que os resultados obtidos através dos testes estivessem mais próximo dos valores verificados
durante as simulações. Um valor expectável para o solo húmido seria um valor entre os 600 e os 700, sendo
um valor superior a 750 seria considerado como solo demasiado húmido.

Dos testes realizados, verificou-se em dois casos diferentes que para um tipo de solo o valor máximo obtido foi
de cerca de 300 e para o caso representado pelo gráfico da figura \ref{fig:graphic} o valor máximo
obtido foi de 400.
Isto deve-se ao facto de nos dois casos, apesar de se estar a realizar testes com o mesmo tipo de plantas,
cada uma das plantas estava plantada em tipos de solo diferentes. A primeira estava plantada num vaso de
pequenas dimensões em que o solo é um composto de turfa loira de Sphagnum \cite{jardinssintra},
como mostra a figura \ref{fig:testconditions}, enquanto que o teste descrito neste artigo,
a planta encontra-se plantada em terra.

\begin{figure}[h]
    \centering
    \includegraphics[width=0.3\textwidth]{rosas-solo-composto.jpg}
    \caption{Plantas no solo de composto Sphagnum}
    \label{fig:testconditions}
\end{figure}

\begin{figure}[h]
    \centering
    \includegraphics[width=0.3\textwidth]{rosa-teste.jpg}
    \caption{Roseira plantada num pequeno jardim}
    \label{fig:roseground}
\end{figure}

Como mencionado no artigo de Abbas et al. \cite{abbas2014smart}, diferentes solos têm diferentes características
sendo uma delas a capacidade de retenção de água. Isto poderá ser um dos fatores, não considerados pelos
discentes, que terá tido influência nos resultados obtidos.

\subsection{Problemas Encontrados}

Durante a realização do projeto descrito neste artigo, surgiram vários problemas, tanto com o hardware
assim como com o software.

A grande maioria dos problemas encontrados com o hardware estavam relacionados com o mau funcionamento dos
dispositivos eletrónicos e com más ligações entre os componentes. A solução para estes problemas foi a
substituição dos mesmo componentes, especialmente alguns cabos.

Os problemas de software estavam relacionados com o editor de texto e o programa do Arduino. Alguns dos problemas
foram resolvidos com a instalação de dependências e bibliotecas para a placa Arduino MKR 1010 wifi.
O problema que atrasou mais o desenvolvimento do projeto é um erro do software Arduino nos quais não era possível
detetar a placa quando estava ligada a uma das entradas USB do computador. Para se encontrar a resolução
consultou-se os vários fóruns, em especial, o fórum da comunidade Arduino. Uma das possíveis soluções era a
reinstalação do software bem como todas as dependências e bibliotecas necessárias. Outra solução é colocar a
placa Arduino em modo bootloader. \cite{arduinoport}

\section{Conclusão e Trabalho Futuro}

Um dos maiores problemas que vamos enfrentar num futuro próximo é a escassez de água potável,
e sendo a água um bem essencial para a sobrevivência humana, existe uma preocupação
em criar sistemas que possibilitem uma melhor gestão da água. Por isto mesmo
é que propomos este sistema de rega de água inteligente. Apesar de não ser um
sistema muito robusto e que pela sua natureza funciona melhor em pequena escala, este
poderia ser uma ponte para a criação de um sistema mais robusto que funciona-se bem
tanto áreas pequenas como jardins ou estufas, como também em áreas maiores como campos agrícolas.

Para trabalho futuro poderia ser integrado um Raspberry Pi neste sistema para uma melhor
gestão de informação. Um dos pontos fracos que o sistema proposto tem é não
ter em conta a previsão do tempo o que pode tornar este sistema menos eficiente.
Esta previsão de tempo poderia ser integrada no sistema através de um API que
envia os dados de previsão em tempo real e juntamente com os dados do sensor de
humidade o sistema tomaria a decisão se regava as plantas ou não.

Este sistema, tal como os sistemas convencionais, também poderia funcionar em horários definidos,
visto que o sistema proposto apenas utiliza a humidade do solo, numa estação de maior calor, a humidade do solo
irá se reduzir a um ritmo mais acelerado do que numa estação onde as temperaturas são mais baixas. Desta forma,
seria possível poupar ainda mais água, bem como energia elétrica, visto que o sistema apenas estaria ligado apenas
nas horas indicadas. Sendo que o sensor de humidade do solo, irá se degradar com o tempo, também ajudaria a
longevidade do sistema.

Não foi possível implementar uma interface gráfica para monitorização do sistema, como era pretendido.
O Arduino MKR 1010 wifi foi o micro controlador escolhido pelo facto de ter capacidade de ligação à Internet.
Desta forma, os dados recolhidos poderiam ser enviados para a Arduino Cloud onde o utilizador do sistema poderia,
então, monitorizar o sistema através do seu smartphone. No entanto, como já referido, não foi possível
implementar tal interface.

Diferentes solos têm características diferentes e devido a estas diferenças, existem outros
fatores a considerar para um sistema de rega inteligente, tais como a capacidade de retenção de água do solo,
a estação do ano, mas também outros fatores como o tipo de plantas para o qual o sistema será utilizado.

\bibliographystyle{IEEEtran}
\bibliography{references}

\end{document}


