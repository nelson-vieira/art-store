\documentclass[conference]{IEEEtran}
\IEEEoverridecommandlockouts
% The preceding line is only needed to identify funding in the first footnote. If that is unneeded, please comment it out.
\usepackage{cite}
\usepackage{amsmath,amssymb,amsfonts}
\usepackage{algorithmic}
\usepackage{textcomp}
\usepackage{xcolor}
\usepackage{colortbl}
\usepackage[rightcaption]{sidecap}
\usepackage{wrapfig}
\usepackage{adjustbox}
\usepackage{siunitx}
\usepackage{hyperref}
\def\UrlBreaks{\do\/\do-}
\usepackage{graphicx} %package to manage images
\graphicspath{ {./images/} }

\def\BibTeX{{\rm B\kern-.05em{\sc i\kern-.025em b}\kern-.08em
    T\kern-.1667em\lower.7ex\hbox{E}\kern-.125emX}}

\makeatletter
\newcommand{\linebreakand}{%
\end{@IEEEauthorhalign}
\hfill\mbox{}\par
\mbox{}\hfill\begin{@IEEEauthorhalign}
}
\makeatother

\begin{document}

\title{Online Store Application: Art Store+}

\author{\IEEEauthorblockN{1\textsuperscript{st} Tomás Marcos}
\IEEEauthorblockA{\textit{Faculdade de Ciências Exatas e da Engenharia} \\
\textit{Universidade da Madeira}\\
Funchal, Portugal \\
2037017@student.uma.pt}
\and
\IEEEauthorblockN{2\textsuperscript{nd} Nelson Vieira}
\IEEEauthorblockA{\textit{Faculdade de Ciências Exatas e da Engenharia} \\
\textit{Universidade da Madeira}\\
Funchal, Portugal \\
2080511@student.uma.pt}
\linebreakand
\IEEEauthorblockN{3\textsuperscript{rd} Luís Olim}
\IEEEauthorblockA{\textit{Faculdade de Ciências Exatas e da Engenharia} \\
\textit{Universidade da Madeira}\\
Funchal, Portugal \\
2034717@student.uma.pt}
}

\maketitle

\begin{abstract}
Utilizar arte para decorar espaços, tanto privados, como uma casa ou um 
restaurante, ou público, como uma praça, não é algo novo. 
Existem vários leilões para compra de arte, mas estes normalmente 
não são acessíveis a todos devido aos preços elevados das obras. 
Isto normalmente leva a que a melhor opção 
para compra de arte seja em lojas de decoração, mas esta arte costuma ser algo 
produzido em massa. Desenvolvemos uma aplicação que pretende 
resolver estes problemas, oferecendo uma loja fácil de aceder onde pessoas 
do público geral possam comprar artigos de arte a preços acessíveis, mas 
também uma plataforma que permita a artistas locais e pouco conhecidos a 
arrancarem a sua carreira e criarem um nome para si mesmos.
\end{abstract}

\begin{IEEEkeywords}
Flutter, dart, aplicação móvel, venda online, loja de arte
\end{IEEEkeywords}

\section{Introdução}

\IEEEPARstart{A}{}rte, do latin \textit{ars} que significa técnica e/ou 
habilidade, pode ser entendida como a atividade humana ligada às 
manifestações de ordem estética ou comunicativa, realizada 
por meio de uma grande variedade de linguagens, tais como: 
arquitetura, desenho, escultura, 
pintura, escrita, música, dança, teatro e cinema, em suas 
variadas combinações. \cite{wikiarte}

"Desde sempre a arte funcionou como espelho da sociedade, 
acompanhando cada contemporaneidade ao longo dos tempos, 
deixando registado na sua obra os diferentes 
momentos vividos, com intensidade e questionamento, como 
um diário pessoal da humanidade, 
desde que o Homem se pensou como tal." \cite{patrimonio}

Como referido, arte pode ter várias formas e é bastante comum 
uma pessoa decorar a sua casa com objetos artísticos, como 
quadros ou estátuas. Normalmente este tipo de arte é encontrado 
em lojas mas são apenas itens produzidos em massa. Isto cria um 
problema para quem gosta de ter algo único. Não é fácil encontrar 
um lugar para arranjar artigos únicos atualmente. Maior parte 
desta arte é vendida por leilões, mas estes leilões costumam 
vender arte de artistas já conhecidos, o que leva a preços 
bastante elevados, algo que maior parte das pessoas não consegue 
comprar. Isto em si também torna difícil de novos talentos emergirem 
e exporem a sua arte, sendo que maior parte dos artistas novos/menos 
conehcidos vendem a sua arte em feiras ou pelas redes sociais ou 
então acabam por doar para exposições, nunca recebendo exposição 
merecida.

É com isto que decidimos criar esta aplicação, Art Store+. Com 
esta aplicação pretendemos criar uma loja online onde artistas 
podem expor as suas criações e começar uma carreira e também 
pessoas do público geral podem procurar e comprar arte para 
decoração ou coleção a preços acessíveis.


\subsection{Formas de arte}
As artes são, muitas vezes, divididas em categorias específicas, 
tais como artes decorativas, 
artes plásticas ou visuais, artes do espectáculo, ou literatura.
As obras de arte que poderiam fazer parte desta aplicação 
são dos mesmos tipos de obras que podem ser vendidas em leilão, 
vendidas em feiras, em exposições ou obras que de alguma 
forma os artistas queiram vender, sendo a única condição que tenham 
alguma forma física, portanto obras de arte digitais não seriam 
aceites para venda nesta aplicação, como é o caso de NFTs.

\subsection{Objetivos}
\begin{itemize}
    \item Tornar obras de arte mais acessíveis ao público geral;
    \item Facilitar a ligação entre artistas e apreciadores de arte;
    \item Fazer com que os artistas possam ganhar a vida com a sua arte, tornando fácil a venda da arte que criam;
    \item Reconhecimento de artistas menos conhecidos;
    \item Promover talento local.
\end{itemize}

\section{Trabalhos Relacionados}

Após feita uma pesquisa de trabalhos relacionados percebemos 
que não existem muitos trabalhos que tenham como foco 
principal uma loja de vendas online de obras de arte, apesar de existirem vários 
trabalhos de aplicações de vendas online. Também existem no mercado algumas 
aplicações de venda de arte online como Artsy \cite{artsy} ou 1000 Museums \cite{1000museums}.

Utilizando os termos "online art store", "art store", "art store application", 
"art store app" no Google Scholar só encontramos dois trabalhos relevante, uma tese 
de licenciatura realizada por Nissilä \cite{nissila2021creating} e outra 
tese de licenciatura realizada por Giam \cite{giam2017setting}, estas 
teses têm como foco a venda de cópias de obras de arte e ambas são 
implementadas usando Squarespace que é uma framework que oferece templates 
de forma a ser fácil a implementação de uma página web. 
Estes trabalhos diferem dos objectivos que traçamos que passa por vender 
obras de arte originais e físicas de artistas menos conhecidos e não de cópias 
de obras de arte.

\section{Métodos e Metadologias}

Para este projeto foi-nos proposto desenvolver uma aplicação móvel que 
permita visualizar e navegar conteúdo, realizar um trabalho de pesquisa 
sobre funcionalidades básicas que a aplicação deve conter, e realizar testes 
de usabilidade. Uma condição para a criação desta aplicação era a utilização 
do \textit{kit} de desenvolvimento de software Flutter.

\subsection{Tecnologias utilizadas}

A aplicação que propomos, é uma aplicação que foi desenvolvida utilizando:

\begin{itemize}
    \item Figma - software de prototipagem;
    \item FlutterFlow - aplicação web para desenvolvimento de aplicações à base de Flutter;
    \item Flutter - linguagem para desenvolvimento de aplicações;
    \item Firebase - plataforma para desenvolvimento de aplicações, atualmente pertencente à Google;
    \item Firestore - base de dados do Firebase;
    \item Visual Studio Code - ambiente de desenvolvimento utilizado para escrita do código;
    \item Git - plataforma de controlo de versões;
    \item GitHub - plataforma de controlo de versões online para trabalho em equipa.
\end{itemize}

Primeiramente, foram realizados alguns esboços sobre qual poderia ser a aparência da aplicação. A figura \ref{fig:sketches} 
mostra os esboços realizados para a aplicação.

\begin{figure}[ht]
    \centering
    \includegraphics[width=0.4\textwidth]{appsketches.png}
    \caption{Esboços da aplicação}
    \label{fig:sketches}
\end{figure}

Desenhou-se um esquema representativo da navegação na aplicação, como mostra a figura \ref{fig:navmap}.

\begin{figure}[ht]
    \centering
    \includegraphics[width=0.4\textwidth]{artstore+map.drawio.png}
    \caption{Mapa de Navigação}
    \label{fig:navmap}
\end{figure}

Como se pode observar, pelo mapa, o utilizador ao iniciar a aplicação será capaz de realizar Login ou Registo. 
Após feito o Login, o utilizador será redirecionado para a Home Page. A partir da Home Page, o utilizador poderá 
navegar para outras páginas, tais como verificar o seu perfil, verificar o carrinho e a sua Wishlist. Também, 
será na Home Page que serão expostas as obras de arte, nos quais o utilizador poderá visitar a página de cada obra. 

O utilizador poderá concluir a compra das obras na página de Checkout, que pode ser acedida pelo Carrinho.

A partir dos esboços criados, procedeu-se à realização de um protótipo no Figma, de maior fidelidade, tal que 
fosse possível testar algumas funcionalidades da aplicação. A este protótipo foram adicionadas mais features do 
que as primeiramente planeadas, tais como a página de pesquisa e focar a imagem da obra.

Após a criação do protótipo utilizando o Figma, procedeu-se a realizar alguns testes, de forma a se obter feedback. 
Com o feedback obtido, foi possível fazer alterações no protótipo realizado através do FlutterFlow, tais como 
alterar o layout da página da obra, de forma a se dar mais destaque à obra. Esta alteração pode ser observada na 
figura \ref{fig:earlyvslate}. À esquerda encontra-se a versão do primeiro protótipo e à direita 
encontra-se a versão final.

\begin{figure}[h]
    \centering
    \includegraphics[width=0.4\textwidth]{artpiece-early-vs-latest.png}
    \caption{Comparação da página da obra de arte}
    \label{fig:earlyvslate}
\end{figure}

Outra alteração significativa é a página de perfil do utilizador. No primeiro protótipo, 
não foi dada muita atenção a esta página, mas após algum feedback dos utilizadores, procedeu-se 
a realizar as alterações que podem ser observadas na figura \ref{fig:profilevs}. Na esquerda, 
está a versão do primeiro protótipo e na direita observa-se a versão final. Acrescentou-se 
os contactos do utilizador, bem como as operações Mudar palavra-passe, Editar perfil e Logout.

\begin{figure}[h]
    \centering
    \includegraphics[width=0.4\textwidth]{profile-early-vs-latest.png}
    \caption{Comparação da página de perfil}
    \label{fig:profilevs}
\end{figure}

\subsection{Testes}

Relativamente aos testes da aplicação, testámos utilizando uma versão de produção da app. 
Realizámos os testes com sete utilizadores e depois dos testes foi pedido a estes que 
realizassem um questionário relativamente à usabilidade da aplicação. 
Dois dos sete utilizadores não quiseram responder ao questionário, 
porque não queriam o seu nome em nenhum registo apesar de explicarmos que 
não iríamos guardar informação sobre estes. Apesar disto tirámos proveito 
dos testes de cada utilizador através do método think aloud. Demos um conjunto de 
tarefas para os utilizadores realizarem, sendo estas:

\begin{itemize}
    \item Registar e fazer login;
    \item Localizar uma certa obra(diferente para cada utilizador) na página principal e comprar essa peça;
    \item Verificar o perfil e editar o perfil;
    \item Verificar o carrinho de compras;
\end{itemize}

Estes testes tinham como objetivo principal avaliar a usabilidade e navegação da aplicação. 
Relativamente a estes obtivemos resultados maioritariamente positivos. 
Os utilizadores acharam todos que a app era fácil de navegar e também gostaram da maneira 
que a informação sobre as peças de arte era disponibilizada. A figura
\ref{fig:navegacaoTest} mostra os resultados do questionário relativamente 
à facilidade dos utilizadores a navegar na aplicação e a figura
\ref{fig:facilitaTest}mostra os resultados relativamente a se os utilizadores 
acham que esta aplicação facilitaria o processo de compra de arte.

\begin{figure}[ht]
    \centering
    \includegraphics[width=0.5\textwidth]{questionarioNavegacao.png}
    \caption{Resultados do questionário para a navegação na app}
    \label{fig:navegacaoTest}
\end{figure}

\begin{figure}[ht]
    \centering
    \includegraphics[width=0.5\textwidth]{questionarioFacilitar.png}
    \caption{Resultados do questionário para a navegação na app}
    \label{fig:facilitaTest}
\end{figure}

Como podemos ver pelos resultados, os utilizadores tiverem muita facilidade a navegar 
pela a aplicação e acharam que esta aplicação iria de facto facilitar o processo de compra de arte.

O ponto mais negativo mencionado pelos utilizadores foi a falta de opções de filtragem/procura. 
Para a versão utilizada para testes ainda não tínhamos implementado estas funcionalidades, 
mas já era algo planeado implementar na aplicação.

Outra observação foi o facto de na página principal não mostrar os preços das obras 
antes de irmos para a página da obra. Não tínhamos pensado nisto inicialmente, 
mas é algo que faz sentido, pois lojas online normalmente têm o preço do item a ser 
mostrado sem que a pessoa tenha que clicar neste e abrir outra página. 
Isto foi implementado depois dos testes.

\subsection{Problemas Encontrados}

Durante a realização do projeto descrito neste artigo, surgiram vários problemas. Inicialmente, tornou-se difícil 
encontrar utilizadores para testar o protótipo inicial, devido à indisponibilidade dos utilizadores de teste. 

Outro problema está relacionado com a criação do protótipo no FlutterFlow 
e a utilização do Firebase e Firestore para gerar o contéudo de forma dinâmica. 
Após se ter criado o protótipo, houve uma tentativa de utilizar a base de dados para 
gerar o conteúdo de forma dinâmica. Isto iria possibilitar ter mais obras em disposição, 
bem como iria permitir realizar pesquisa na aplicação, filtrando as obras por nome, 
tipo de obra, artista ou detalhes da obra.

Devido às dificuldades encontradas com a integração da base de dados, 
não foi possível implementar a página de pesquisa, a página de checkout não 
recebe inputs e não é possível realizar pesquisa na página de pesquisa.


\section{Conclusão e Trabalho Futuro}

\bibliographystyle{IEEEtran}
\bibliography{references}

\end{document}


